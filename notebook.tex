
% Default to the notebook output style

    


% Inherit from the specified cell style.




    
\documentclass[11pt]{article}

    
    
    \usepackage[T1]{fontenc}
    % Nicer default font (+ math font) than Computer Modern for most use cases
    \usepackage{mathpazo}

    % Basic figure setup, for now with no caption control since it's done
    % automatically by Pandoc (which extracts ![](path) syntax from Markdown).
    \usepackage{graphicx}
    % We will generate all images so they have a width \maxwidth. This means
    % that they will get their normal width if they fit onto the page, but
    % are scaled down if they would overflow the margins.
    \makeatletter
    \def\maxwidth{\ifdim\Gin@nat@width>\linewidth\linewidth
    \else\Gin@nat@width\fi}
    \makeatother
    \let\Oldincludegraphics\includegraphics
    % Set max figure width to be 80% of text width, for now hardcoded.
    \renewcommand{\includegraphics}[1]{\Oldincludegraphics[width=.8\maxwidth]{#1}}
    % Ensure that by default, figures have no caption (until we provide a
    % proper Figure object with a Caption API and a way to capture that
    % in the conversion process - todo).
    \usepackage{caption}
    \DeclareCaptionLabelFormat{nolabel}{}
    \captionsetup{labelformat=nolabel}

    \usepackage{adjustbox} % Used to constrain images to a maximum size 
    \usepackage{xcolor} % Allow colors to be defined
    \usepackage{enumerate} % Needed for markdown enumerations to work
    \usepackage{geometry} % Used to adjust the document margins
    \usepackage{amsmath} % Equations
    \usepackage{amssymb} % Equations
    \usepackage{textcomp} % defines textquotesingle
    % Hack from http://tex.stackexchange.com/a/47451/13684:
    \AtBeginDocument{%
        \def\PYZsq{\textquotesingle}% Upright quotes in Pygmentized code
    }
    \usepackage{upquote} % Upright quotes for verbatim code
    \usepackage{eurosym} % defines \euro
    \usepackage[mathletters]{ucs} % Extended unicode (utf-8) support
    \usepackage[utf8x]{inputenc} % Allow utf-8 characters in the tex document
    \usepackage{fancyvrb} % verbatim replacement that allows latex
    \usepackage{grffile} % extends the file name processing of package graphics 
                         % to support a larger range 
    % The hyperref package gives us a pdf with properly built
    % internal navigation ('pdf bookmarks' for the table of contents,
    % internal cross-reference links, web links for URLs, etc.)
    \usepackage{hyperref}
    \usepackage{longtable} % longtable support required by pandoc >1.10
    \usepackage{booktabs}  % table support for pandoc > 1.12.2
    \usepackage[inline]{enumitem} % IRkernel/repr support (it uses the enumerate* environment)
    \usepackage[normalem]{ulem} % ulem is needed to support strikethroughs (\sout)
                                % normalem makes italics be italics, not underlines
    

    
    
    % Colors for the hyperref package
    \definecolor{urlcolor}{rgb}{0,.145,.698}
    \definecolor{linkcolor}{rgb}{.71,0.21,0.01}
    \definecolor{citecolor}{rgb}{.12,.54,.11}

    % ANSI colors
    \definecolor{ansi-black}{HTML}{3E424D}
    \definecolor{ansi-black-intense}{HTML}{282C36}
    \definecolor{ansi-red}{HTML}{E75C58}
    \definecolor{ansi-red-intense}{HTML}{B22B31}
    \definecolor{ansi-green}{HTML}{00A250}
    \definecolor{ansi-green-intense}{HTML}{007427}
    \definecolor{ansi-yellow}{HTML}{DDB62B}
    \definecolor{ansi-yellow-intense}{HTML}{B27D12}
    \definecolor{ansi-blue}{HTML}{208FFB}
    \definecolor{ansi-blue-intense}{HTML}{0065CA}
    \definecolor{ansi-magenta}{HTML}{D160C4}
    \definecolor{ansi-magenta-intense}{HTML}{A03196}
    \definecolor{ansi-cyan}{HTML}{60C6C8}
    \definecolor{ansi-cyan-intense}{HTML}{258F8F}
    \definecolor{ansi-white}{HTML}{C5C1B4}
    \definecolor{ansi-white-intense}{HTML}{A1A6B2}

    % commands and environments needed by pandoc snippets
    % extracted from the output of `pandoc -s`
    \providecommand{\tightlist}{%
      \setlength{\itemsep}{0pt}\setlength{\parskip}{0pt}}
    \DefineVerbatimEnvironment{Highlighting}{Verbatim}{commandchars=\\\{\}}
    % Add ',fontsize=\small' for more characters per line
    \newenvironment{Shaded}{}{}
    \newcommand{\KeywordTok}[1]{\textcolor[rgb]{0.00,0.44,0.13}{\textbf{{#1}}}}
    \newcommand{\DataTypeTok}[1]{\textcolor[rgb]{0.56,0.13,0.00}{{#1}}}
    \newcommand{\DecValTok}[1]{\textcolor[rgb]{0.25,0.63,0.44}{{#1}}}
    \newcommand{\BaseNTok}[1]{\textcolor[rgb]{0.25,0.63,0.44}{{#1}}}
    \newcommand{\FloatTok}[1]{\textcolor[rgb]{0.25,0.63,0.44}{{#1}}}
    \newcommand{\CharTok}[1]{\textcolor[rgb]{0.25,0.44,0.63}{{#1}}}
    \newcommand{\StringTok}[1]{\textcolor[rgb]{0.25,0.44,0.63}{{#1}}}
    \newcommand{\CommentTok}[1]{\textcolor[rgb]{0.38,0.63,0.69}{\textit{{#1}}}}
    \newcommand{\OtherTok}[1]{\textcolor[rgb]{0.00,0.44,0.13}{{#1}}}
    \newcommand{\AlertTok}[1]{\textcolor[rgb]{1.00,0.00,0.00}{\textbf{{#1}}}}
    \newcommand{\FunctionTok}[1]{\textcolor[rgb]{0.02,0.16,0.49}{{#1}}}
    \newcommand{\RegionMarkerTok}[1]{{#1}}
    \newcommand{\ErrorTok}[1]{\textcolor[rgb]{1.00,0.00,0.00}{\textbf{{#1}}}}
    \newcommand{\NormalTok}[1]{{#1}}
    
    % Additional commands for more recent versions of Pandoc
    \newcommand{\ConstantTok}[1]{\textcolor[rgb]{0.53,0.00,0.00}{{#1}}}
    \newcommand{\SpecialCharTok}[1]{\textcolor[rgb]{0.25,0.44,0.63}{{#1}}}
    \newcommand{\VerbatimStringTok}[1]{\textcolor[rgb]{0.25,0.44,0.63}{{#1}}}
    \newcommand{\SpecialStringTok}[1]{\textcolor[rgb]{0.73,0.40,0.53}{{#1}}}
    \newcommand{\ImportTok}[1]{{#1}}
    \newcommand{\DocumentationTok}[1]{\textcolor[rgb]{0.73,0.13,0.13}{\textit{{#1}}}}
    \newcommand{\AnnotationTok}[1]{\textcolor[rgb]{0.38,0.63,0.69}{\textbf{\textit{{#1}}}}}
    \newcommand{\CommentVarTok}[1]{\textcolor[rgb]{0.38,0.63,0.69}{\textbf{\textit{{#1}}}}}
    \newcommand{\VariableTok}[1]{\textcolor[rgb]{0.10,0.09,0.49}{{#1}}}
    \newcommand{\ControlFlowTok}[1]{\textcolor[rgb]{0.00,0.44,0.13}{\textbf{{#1}}}}
    \newcommand{\OperatorTok}[1]{\textcolor[rgb]{0.40,0.40,0.40}{{#1}}}
    \newcommand{\BuiltInTok}[1]{{#1}}
    \newcommand{\ExtensionTok}[1]{{#1}}
    \newcommand{\PreprocessorTok}[1]{\textcolor[rgb]{0.74,0.48,0.00}{{#1}}}
    \newcommand{\AttributeTok}[1]{\textcolor[rgb]{0.49,0.56,0.16}{{#1}}}
    \newcommand{\InformationTok}[1]{\textcolor[rgb]{0.38,0.63,0.69}{\textbf{\textit{{#1}}}}}
    \newcommand{\WarningTok}[1]{\textcolor[rgb]{0.38,0.63,0.69}{\textbf{\textit{{#1}}}}}
    
    
    % Define a nice break command that doesn't care if a line doesn't already
    % exist.
    \def\br{\hspace*{\fill} \\* }
    % Math Jax compatability definitions
    \def\gt{>}
    \def\lt{<}
    % Document parameters
    \title{numeric\_data\_characterization}
    
    
    

    % Pygments definitions
    
\makeatletter
\def\PY@reset{\let\PY@it=\relax \let\PY@bf=\relax%
    \let\PY@ul=\relax \let\PY@tc=\relax%
    \let\PY@bc=\relax \let\PY@ff=\relax}
\def\PY@tok#1{\csname PY@tok@#1\endcsname}
\def\PY@toks#1+{\ifx\relax#1\empty\else%
    \PY@tok{#1}\expandafter\PY@toks\fi}
\def\PY@do#1{\PY@bc{\PY@tc{\PY@ul{%
    \PY@it{\PY@bf{\PY@ff{#1}}}}}}}
\def\PY#1#2{\PY@reset\PY@toks#1+\relax+\PY@do{#2}}

\expandafter\def\csname PY@tok@il\endcsname{\def\PY@tc##1{\textcolor[rgb]{0.40,0.40,0.40}{##1}}}
\expandafter\def\csname PY@tok@kr\endcsname{\let\PY@bf=\textbf\def\PY@tc##1{\textcolor[rgb]{0.00,0.50,0.00}{##1}}}
\expandafter\def\csname PY@tok@nl\endcsname{\def\PY@tc##1{\textcolor[rgb]{0.63,0.63,0.00}{##1}}}
\expandafter\def\csname PY@tok@cs\endcsname{\let\PY@it=\textit\def\PY@tc##1{\textcolor[rgb]{0.25,0.50,0.50}{##1}}}
\expandafter\def\csname PY@tok@gu\endcsname{\let\PY@bf=\textbf\def\PY@tc##1{\textcolor[rgb]{0.50,0.00,0.50}{##1}}}
\expandafter\def\csname PY@tok@kc\endcsname{\let\PY@bf=\textbf\def\PY@tc##1{\textcolor[rgb]{0.00,0.50,0.00}{##1}}}
\expandafter\def\csname PY@tok@gp\endcsname{\let\PY@bf=\textbf\def\PY@tc##1{\textcolor[rgb]{0.00,0.00,0.50}{##1}}}
\expandafter\def\csname PY@tok@sh\endcsname{\def\PY@tc##1{\textcolor[rgb]{0.73,0.13,0.13}{##1}}}
\expandafter\def\csname PY@tok@ne\endcsname{\let\PY@bf=\textbf\def\PY@tc##1{\textcolor[rgb]{0.82,0.25,0.23}{##1}}}
\expandafter\def\csname PY@tok@kt\endcsname{\def\PY@tc##1{\textcolor[rgb]{0.69,0.00,0.25}{##1}}}
\expandafter\def\csname PY@tok@s1\endcsname{\def\PY@tc##1{\textcolor[rgb]{0.73,0.13,0.13}{##1}}}
\expandafter\def\csname PY@tok@sd\endcsname{\let\PY@it=\textit\def\PY@tc##1{\textcolor[rgb]{0.73,0.13,0.13}{##1}}}
\expandafter\def\csname PY@tok@vg\endcsname{\def\PY@tc##1{\textcolor[rgb]{0.10,0.09,0.49}{##1}}}
\expandafter\def\csname PY@tok@mh\endcsname{\def\PY@tc##1{\textcolor[rgb]{0.40,0.40,0.40}{##1}}}
\expandafter\def\csname PY@tok@ow\endcsname{\let\PY@bf=\textbf\def\PY@tc##1{\textcolor[rgb]{0.67,0.13,1.00}{##1}}}
\expandafter\def\csname PY@tok@nt\endcsname{\let\PY@bf=\textbf\def\PY@tc##1{\textcolor[rgb]{0.00,0.50,0.00}{##1}}}
\expandafter\def\csname PY@tok@mb\endcsname{\def\PY@tc##1{\textcolor[rgb]{0.40,0.40,0.40}{##1}}}
\expandafter\def\csname PY@tok@vm\endcsname{\def\PY@tc##1{\textcolor[rgb]{0.10,0.09,0.49}{##1}}}
\expandafter\def\csname PY@tok@mo\endcsname{\def\PY@tc##1{\textcolor[rgb]{0.40,0.40,0.40}{##1}}}
\expandafter\def\csname PY@tok@ge\endcsname{\let\PY@it=\textit}
\expandafter\def\csname PY@tok@sa\endcsname{\def\PY@tc##1{\textcolor[rgb]{0.73,0.13,0.13}{##1}}}
\expandafter\def\csname PY@tok@ss\endcsname{\def\PY@tc##1{\textcolor[rgb]{0.10,0.09,0.49}{##1}}}
\expandafter\def\csname PY@tok@o\endcsname{\def\PY@tc##1{\textcolor[rgb]{0.40,0.40,0.40}{##1}}}
\expandafter\def\csname PY@tok@err\endcsname{\def\PY@bc##1{\setlength{\fboxsep}{0pt}\fcolorbox[rgb]{1.00,0.00,0.00}{1,1,1}{\strut ##1}}}
\expandafter\def\csname PY@tok@mi\endcsname{\def\PY@tc##1{\textcolor[rgb]{0.40,0.40,0.40}{##1}}}
\expandafter\def\csname PY@tok@s2\endcsname{\def\PY@tc##1{\textcolor[rgb]{0.73,0.13,0.13}{##1}}}
\expandafter\def\csname PY@tok@fm\endcsname{\def\PY@tc##1{\textcolor[rgb]{0.00,0.00,1.00}{##1}}}
\expandafter\def\csname PY@tok@dl\endcsname{\def\PY@tc##1{\textcolor[rgb]{0.73,0.13,0.13}{##1}}}
\expandafter\def\csname PY@tok@nn\endcsname{\let\PY@bf=\textbf\def\PY@tc##1{\textcolor[rgb]{0.00,0.00,1.00}{##1}}}
\expandafter\def\csname PY@tok@w\endcsname{\def\PY@tc##1{\textcolor[rgb]{0.73,0.73,0.73}{##1}}}
\expandafter\def\csname PY@tok@si\endcsname{\let\PY@bf=\textbf\def\PY@tc##1{\textcolor[rgb]{0.73,0.40,0.53}{##1}}}
\expandafter\def\csname PY@tok@vi\endcsname{\def\PY@tc##1{\textcolor[rgb]{0.10,0.09,0.49}{##1}}}
\expandafter\def\csname PY@tok@gi\endcsname{\def\PY@tc##1{\textcolor[rgb]{0.00,0.63,0.00}{##1}}}
\expandafter\def\csname PY@tok@se\endcsname{\let\PY@bf=\textbf\def\PY@tc##1{\textcolor[rgb]{0.73,0.40,0.13}{##1}}}
\expandafter\def\csname PY@tok@na\endcsname{\def\PY@tc##1{\textcolor[rgb]{0.49,0.56,0.16}{##1}}}
\expandafter\def\csname PY@tok@nb\endcsname{\def\PY@tc##1{\textcolor[rgb]{0.00,0.50,0.00}{##1}}}
\expandafter\def\csname PY@tok@nd\endcsname{\def\PY@tc##1{\textcolor[rgb]{0.67,0.13,1.00}{##1}}}
\expandafter\def\csname PY@tok@ni\endcsname{\let\PY@bf=\textbf\def\PY@tc##1{\textcolor[rgb]{0.60,0.60,0.60}{##1}}}
\expandafter\def\csname PY@tok@k\endcsname{\let\PY@bf=\textbf\def\PY@tc##1{\textcolor[rgb]{0.00,0.50,0.00}{##1}}}
\expandafter\def\csname PY@tok@ch\endcsname{\let\PY@it=\textit\def\PY@tc##1{\textcolor[rgb]{0.25,0.50,0.50}{##1}}}
\expandafter\def\csname PY@tok@sb\endcsname{\def\PY@tc##1{\textcolor[rgb]{0.73,0.13,0.13}{##1}}}
\expandafter\def\csname PY@tok@vc\endcsname{\def\PY@tc##1{\textcolor[rgb]{0.10,0.09,0.49}{##1}}}
\expandafter\def\csname PY@tok@gh\endcsname{\let\PY@bf=\textbf\def\PY@tc##1{\textcolor[rgb]{0.00,0.00,0.50}{##1}}}
\expandafter\def\csname PY@tok@sx\endcsname{\def\PY@tc##1{\textcolor[rgb]{0.00,0.50,0.00}{##1}}}
\expandafter\def\csname PY@tok@go\endcsname{\def\PY@tc##1{\textcolor[rgb]{0.53,0.53,0.53}{##1}}}
\expandafter\def\csname PY@tok@gd\endcsname{\def\PY@tc##1{\textcolor[rgb]{0.63,0.00,0.00}{##1}}}
\expandafter\def\csname PY@tok@gr\endcsname{\def\PY@tc##1{\textcolor[rgb]{1.00,0.00,0.00}{##1}}}
\expandafter\def\csname PY@tok@cm\endcsname{\let\PY@it=\textit\def\PY@tc##1{\textcolor[rgb]{0.25,0.50,0.50}{##1}}}
\expandafter\def\csname PY@tok@gs\endcsname{\let\PY@bf=\textbf}
\expandafter\def\csname PY@tok@mf\endcsname{\def\PY@tc##1{\textcolor[rgb]{0.40,0.40,0.40}{##1}}}
\expandafter\def\csname PY@tok@cp\endcsname{\def\PY@tc##1{\textcolor[rgb]{0.74,0.48,0.00}{##1}}}
\expandafter\def\csname PY@tok@bp\endcsname{\def\PY@tc##1{\textcolor[rgb]{0.00,0.50,0.00}{##1}}}
\expandafter\def\csname PY@tok@nc\endcsname{\let\PY@bf=\textbf\def\PY@tc##1{\textcolor[rgb]{0.00,0.00,1.00}{##1}}}
\expandafter\def\csname PY@tok@s\endcsname{\def\PY@tc##1{\textcolor[rgb]{0.73,0.13,0.13}{##1}}}
\expandafter\def\csname PY@tok@kn\endcsname{\let\PY@bf=\textbf\def\PY@tc##1{\textcolor[rgb]{0.00,0.50,0.00}{##1}}}
\expandafter\def\csname PY@tok@nf\endcsname{\def\PY@tc##1{\textcolor[rgb]{0.00,0.00,1.00}{##1}}}
\expandafter\def\csname PY@tok@sc\endcsname{\def\PY@tc##1{\textcolor[rgb]{0.73,0.13,0.13}{##1}}}
\expandafter\def\csname PY@tok@kd\endcsname{\let\PY@bf=\textbf\def\PY@tc##1{\textcolor[rgb]{0.00,0.50,0.00}{##1}}}
\expandafter\def\csname PY@tok@c\endcsname{\let\PY@it=\textit\def\PY@tc##1{\textcolor[rgb]{0.25,0.50,0.50}{##1}}}
\expandafter\def\csname PY@tok@m\endcsname{\def\PY@tc##1{\textcolor[rgb]{0.40,0.40,0.40}{##1}}}
\expandafter\def\csname PY@tok@no\endcsname{\def\PY@tc##1{\textcolor[rgb]{0.53,0.00,0.00}{##1}}}
\expandafter\def\csname PY@tok@c1\endcsname{\let\PY@it=\textit\def\PY@tc##1{\textcolor[rgb]{0.25,0.50,0.50}{##1}}}
\expandafter\def\csname PY@tok@nv\endcsname{\def\PY@tc##1{\textcolor[rgb]{0.10,0.09,0.49}{##1}}}
\expandafter\def\csname PY@tok@sr\endcsname{\def\PY@tc##1{\textcolor[rgb]{0.73,0.40,0.53}{##1}}}
\expandafter\def\csname PY@tok@cpf\endcsname{\let\PY@it=\textit\def\PY@tc##1{\textcolor[rgb]{0.25,0.50,0.50}{##1}}}
\expandafter\def\csname PY@tok@gt\endcsname{\def\PY@tc##1{\textcolor[rgb]{0.00,0.27,0.87}{##1}}}
\expandafter\def\csname PY@tok@kp\endcsname{\def\PY@tc##1{\textcolor[rgb]{0.00,0.50,0.00}{##1}}}

\def\PYZbs{\char`\\}
\def\PYZus{\char`\_}
\def\PYZob{\char`\{}
\def\PYZcb{\char`\}}
\def\PYZca{\char`\^}
\def\PYZam{\char`\&}
\def\PYZlt{\char`\<}
\def\PYZgt{\char`\>}
\def\PYZsh{\char`\#}
\def\PYZpc{\char`\%}
\def\PYZdl{\char`\$}
\def\PYZhy{\char`\-}
\def\PYZsq{\char`\'}
\def\PYZdq{\char`\"}
\def\PYZti{\char`\~}
% for compatibility with earlier versions
\def\PYZat{@}
\def\PYZlb{[}
\def\PYZrb{]}
\makeatother


    % Exact colors from NB
    \definecolor{incolor}{rgb}{0.0, 0.0, 0.5}
    \definecolor{outcolor}{rgb}{0.545, 0.0, 0.0}



    
    % Prevent overflowing lines due to hard-to-break entities
    \sloppy 
    % Setup hyperref package
    \hypersetup{
      breaklinks=true,  % so long urls are correctly broken across lines
      colorlinks=true,
      urlcolor=urlcolor,
      linkcolor=linkcolor,
      citecolor=citecolor,
      }
    % Slightly bigger margins than the latex defaults
    
    \geometry{verbose,tmargin=1in,bmargin=1in,lmargin=1in,rmargin=1in}
    
    

    \begin{document}
    
    
    \maketitle
    
    

    
    DeCART Summer School for Biomedical Data Science

\hypertarget{basic-numeric-data-characterization}{%
\section{Basic Numeric Data
Characterization}\label{basic-numeric-data-characterization}}

Numpy provides a number of functions/methods for characterizing

    \begin{Verbatim}[commandchars=\\\{\}]
{\color{incolor}In [{\color{incolor}1}]:} \PY{k+kn}{import} \PY{n+nn}{os}
        \PY{k+kn}{import} \PY{n+nn}{numpy} \PY{k}{as} \PY{n+nn}{np}
        \PY{k+kn}{import} \PY{n+nn}{utils}
        \PY{k+kn}{import} \PY{n+nn}{numpy}\PY{n+nn}{.}\PY{n+nn}{random} \PY{k}{as} \PY{n+nn}{ra}
        \PY{k+kn}{import} \PY{n+nn}{pandas} \PY{k}{as} \PY{n+nn}{pd}
\end{Verbatim}


    \begin{Verbatim}[commandchars=\\\{\}]
{\color{incolor}In [{\color{incolor}2}]:} \PY{k+kn}{from} \PY{n+nn}{quizzes}\PY{n+nn}{.}\PY{n+nn}{characterizing\PYZus{}numeric\PYZus{}data} \PY{k}{import} \PY{o}{*}
\end{Verbatim}


    \begin{Verbatim}[commandchars=\\\{\}]
{\color{incolor}In [{\color{incolor}3}]:} \PY{n}{pd}\PY{o}{.}\PY{n}{\PYZus{}\PYZus{}version\PYZus{}\PYZus{}}
\end{Verbatim}


\begin{Verbatim}[commandchars=\\\{\}]
{\color{outcolor}Out[{\color{outcolor}3}]:} '0.23.1'
\end{Verbatim}
            
    \begin{Verbatim}[commandchars=\\\{\}]
{\color{incolor}In [{\color{incolor}6}]:} \PY{n}{DATADIR} \PY{o}{=} \PY{n}{os}\PY{o}{.}\PY{n}{path}\PY{o}{.}\PY{n}{join}\PY{p}{(}\PY{n}{os}\PY{o}{.}\PY{n}{path}\PY{o}{.}\PY{n}{expanduser}\PY{p}{(}\PY{l+s+s2}{\PYZdq{}}\PY{l+s+s2}{\PYZti{}}\PY{l+s+s2}{\PYZdq{}}\PY{p}{)}\PY{p}{,}\PY{l+s+s2}{\PYZdq{}}\PY{l+s+s2}{DATA}\PY{l+s+s2}{\PYZdq{}}\PY{p}{)}
        \PY{n}{HRDIR} \PY{o}{=} \PY{n}{os}\PY{o}{.}\PY{n}{path}\PY{o}{.}\PY{n}{join}\PY{p}{(}\PY{n}{DATADIR}\PY{p}{,}\PY{l+s+s2}{\PYZdq{}}\PY{l+s+s2}{Numerics}\PY{l+s+s2}{\PYZdq{}}\PY{p}{,} \PY{l+s+s2}{\PYZdq{}}\PY{l+s+s2}{mimic2}\PY{l+s+s2}{\PYZdq{}}\PY{p}{,} \PY{l+s+s2}{\PYZdq{}}\PY{l+s+s2}{hr}\PY{l+s+s2}{\PYZdq{}}\PY{p}{,} \PY{l+s+s2}{\PYZdq{}}\PY{l+s+s2}{subjects}\PY{l+s+s2}{\PYZdq{}}\PY{p}{)}
        \PY{n}{BPDIR} \PY{o}{=} \PY{n}{os}\PY{o}{.}\PY{n}{path}\PY{o}{.}\PY{n}{join}\PY{p}{(}\PY{n}{DATADIR}\PY{p}{,}\PY{l+s+s2}{\PYZdq{}}\PY{l+s+s2}{Numerics}\PY{l+s+s2}{\PYZdq{}}\PY{p}{,} \PY{l+s+s2}{\PYZdq{}}\PY{l+s+s2}{mimic2}\PY{l+s+s2}{\PYZdq{}}\PY{p}{,} \PY{l+s+s2}{\PYZdq{}}\PY{l+s+s2}{bp}\PY{l+s+s2}{\PYZdq{}}\PY{p}{,} \PY{l+s+s2}{\PYZdq{}}\PY{l+s+s2}{subjects}\PY{l+s+s2}{\PYZdq{}}\PY{p}{)}
        
        \PY{n}{hr\PYZus{}files} \PY{o}{=} \PY{n}{os}\PY{o}{.}\PY{n}{listdir}\PY{p}{(}\PY{n}{HRDIR}\PY{p}{)}
\end{Verbatim}


    \hypertarget{heart-rate-file-for-patient-3325}{%
\subsection{\texorpdfstring{Heart Rate file for patient
\texttt{3325}}{Heart Rate file for patient 3325}}\label{heart-rate-file-for-patient-3325}}

We are going to start with using
\href{http://pandas.pydata.org/}{Pandas} to read, summarize, and
visualize numeric data. We are going to start with the MIMIC2 patient
\texttt{\#3325}. This is what the first five lines of the file looks
like (You can explore this in the Linux shell with less, more, or cat.

\begin{Shaded}
\begin{Highlighting}[]
\DecValTok{88}
\DecValTok{84}
\DecValTok{87}
\DecValTok{78}
\DecValTok{85}
\end{Highlighting}
\end{Shaded}

Pandas has two basic functions for reading in tabular data:
\href{https://pandas.pydata.org/pandas-docs/stable/generated/pandas.read_table.html}{\texttt{read\_table}}
and
{[}\texttt{read\_csv}{]}(https://pandas.pydata.org/pandas-docs/stable/generat

ed/pandas.read\_csv.html). They are really the smae function with
different default values for the delimiter, the character that is used
to separate the data on each row: (a tab (\texttt{\textbackslash{}t})
character for \texttt{read\_table} and a comma (\texttt{,}) for
\texttt{read\_csv}. We will use \texttt{read\_table} to read in the
heart rate values.

Both of these functions swill return a
\href{https://pandas.pydata.org/pandas-docs/stable/generated/pandas.DataFrame.html}{Pandas
DataFrame} which is a \textgreater{}Two-dimensional size-mutable,
potentially heterogeneous tabular data structure with labeled axes (rows
and columns).

Going back to an earlier analogy, a data structure is a complex molecule
of data.

A Pandas DataFrame is an object (everything is an object), and so it has
attributes and methods. Two of the methods we will use right off the bat
are

\begin{itemize}
\tightlist
\item
  \href{https://pandas.pydata.org/pandas-docs/stable/generated/pandas.DataFrame.head.html\#pandas.DataFrame.head}{\texttt{head()}}
\item
  \href{https://pandas.pydata.org/pandas-docs/stable/generated/pandas.DataFrame.tail.html\#pandas.DataFrame.tail}{\texttt{tail()}}
\end{itemize}

which return the first (last) n (default=5) rows of the DataFrame.

\hypertarget{read-in-the-data-and-look-at-the-head}{%
\subsubsection{Read in the data and look at the
head}\label{read-in-the-data-and-look-at-the-head}}

    \begin{Verbatim}[commandchars=\\\{\}]
{\color{incolor}In [{\color{incolor}7}]:} \PY{n}{hr} \PY{o}{=} \PY{n}{pd}\PY{o}{.}\PY{n}{read\PYZus{}table}\PY{p}{(}\PY{n}{os}\PY{o}{.}\PY{n}{path}\PY{o}{.}\PY{n}{join}\PY{p}{(}\PY{n}{HRDIR}\PY{p}{,}\PY{l+s+s1}{\PYZsq{}}\PY{l+s+s1}{3235.txt}\PY{l+s+s1}{\PYZsq{}}\PY{p}{)}\PY{p}{)}
        \PY{n}{hr}\PY{o}{.}\PY{n}{head}\PY{p}{(}\PY{p}{)}
\end{Verbatim}


\begin{Verbatim}[commandchars=\\\{\}]
{\color{outcolor}Out[{\color{outcolor}7}]:}    88
        0  84
        1  87
        2  78
        3  85
        4  80
\end{Verbatim}
            
    The DataFrame provides an \textbf{index} for each row (e.g.
\(0, 1, 2, \cdots\)). It will also display the label (name) for each
column.

\hypertarget{does-anything-seem-wrong}{%
\subsubsection{Does anything seem
wrong?}\label{does-anything-seem-wrong}}

    \begin{Verbatim}[commandchars=\\\{\}]
{\color{incolor}In [{\color{incolor}8}]:} \PY{n}{hr}\PY{o}{.}\PY{n}{shape}\PY{p}{,} \PY{n}{hr}\PY{o}{.}\PY{n}{size}
\end{Verbatim}


\begin{Verbatim}[commandchars=\\\{\}]
{\color{outcolor}Out[{\color{outcolor}8}]:} ((482, 1), 482)
\end{Verbatim}
            
    \hypertarget{reread-the-data-specifying-what-to-use-for-the-header}{%
\subsubsection{Reread the data, specifying what to use for the
header}\label{reread-the-data-specifying-what-to-use-for-the-header}}

    \begin{Verbatim}[commandchars=\\\{\}]
{\color{incolor}In [{\color{incolor}9}]:} \PY{n}{hr} \PY{o}{=} \PY{n}{pd}\PY{o}{.}\PY{n}{read\PYZus{}table}\PY{p}{(}\PY{n}{os}\PY{o}{.}\PY{n}{path}\PY{o}{.}\PY{n}{join}\PY{p}{(}\PY{n}{HRDIR}\PY{p}{,}\PY{l+s+s1}{\PYZsq{}}\PY{l+s+s1}{3235.txt}\PY{l+s+s1}{\PYZsq{}}\PY{p}{)}\PY{p}{,} \PY{n}{header}\PY{o}{=}\PY{k+kc}{None}\PY{p}{)}
        \PY{n}{hr}\PY{o}{.}\PY{n}{head}\PY{p}{(}\PY{p}{)}
\end{Verbatim}


\begin{Verbatim}[commandchars=\\\{\}]
{\color{outcolor}Out[{\color{outcolor}9}]:}     0
        0  88
        1  84
        2  87
        3  78
        4  85
\end{Verbatim}
            
    This looks better but zero for a column label/name is not very
meaningful. We can provide names to use for the columns with a
\texttt{names} keyword argument.

\hypertarget{reread-the-data-providing-a-name-for-the-heart-rate-column}{%
\subsubsection{Reread the data providing a name for the heart rate
column}\label{reread-the-data-providing-a-name-for-the-heart-rate-column}}

    \begin{Verbatim}[commandchars=\\\{\}]
{\color{incolor}In [{\color{incolor}10}]:} \PY{n}{hr} \PY{o}{=} \PY{n}{pd}\PY{o}{.}\PY{n}{read\PYZus{}table}\PY{p}{(}\PY{n}{os}\PY{o}{.}\PY{n}{path}\PY{o}{.}\PY{n}{join}\PY{p}{(}\PY{n}{HRDIR}\PY{p}{,}\PY{l+s+s1}{\PYZsq{}}\PY{l+s+s1}{3235.txt}\PY{l+s+s1}{\PYZsq{}}\PY{p}{)}\PY{p}{,} \PY{n}{header}\PY{o}{=}\PY{k+kc}{None}\PY{p}{,} \PY{n}{names}\PY{o}{=}\PY{p}{[}\PY{l+s+s2}{\PYZdq{}}\PY{l+s+s2}{heart\PYZus{}rate}\PY{l+s+s2}{\PYZdq{}}\PY{p}{,}\PY{p}{]}\PY{p}{)}
         \PY{n}{hr}\PY{o}{.}\PY{n}{head}\PY{p}{(}\PY{p}{)}
\end{Verbatim}


\begin{Verbatim}[commandchars=\\\{\}]
{\color{outcolor}Out[{\color{outcolor}10}]:}    heart\_rate
         0          88
         1          84
         2          87
         3          78
         4          85
\end{Verbatim}
            
    \hypertarget{dataframe-attributes}{%
\subsubsection{DataFrame Attributes}\label{dataframe-attributes}}

DataFrames have attributes that describe it, including

\begin{itemize}
\tightlist
\item
  \texttt{shape}: shape of the frame
\item
  \texttt{size}: total number of elements in the frame
\item
  \texttt{dtypes}: the data types for each column in the frame
\end{itemize}

    \begin{Verbatim}[commandchars=\\\{\}]
{\color{incolor}In [{\color{incolor}11}]:} \PY{n+nb}{print}\PY{p}{(}\PY{n}{hr}\PY{o}{.}\PY{n}{shape}\PY{p}{,} \PY{n}{hr}\PY{o}{.}\PY{n}{size}\PY{p}{)}
         \PY{n+nb}{print}\PY{p}{(}\PY{n}{hr}\PY{o}{.}\PY{n}{dtypes}\PY{p}{)}
\end{Verbatim}


    \begin{Verbatim}[commandchars=\\\{\}]
(483, 1) 483
heart\_rate    int64
dtype: object

    \end{Verbatim}

    \hypertarget{exercise}{%
\subsection{Exercise}\label{exercise}}

Use the Pandas \texttt{read\_table}function to read in the blood
pressure data for the same patient (\texttt{3235}). Answer the following
questions:

\begin{enumerate}
\def\labelenumi{\arabic{enumi}.}
\tightlist
\item
  How many rows are in the blood pressure data frame?
\item
  What data type (e.g.~np.uint8) is used for the first column (systolic)
  of measurements?
\end{enumerate}

    \begin{Verbatim}[commandchars=\\\{\}]
{\color{incolor}In [{\color{incolor}13}]:} \PY{n}{data\PYZus{}shape}\PY{p}{(}\PY{l+m+mi}{485}\PY{p}{)}
\end{Verbatim}


\begin{Verbatim}[commandchars=\\\{\}]
{\color{outcolor}Out[{\color{outcolor}13}]:} 'You provided the correct number of rows'
\end{Verbatim}
            
    \begin{Verbatim}[commandchars=\\\{\}]
{\color{incolor}In [{\color{incolor}24}]:} \PY{n}{bp} \PY{o}{=} \PY{n}{pd}\PY{o}{.}\PY{n}{read\PYZus{}table}\PY{p}{(}\PY{n}{os}\PY{o}{.}\PY{n}{path}\PY{o}{.}\PY{n}{join}\PY{p}{(}\PY{n}{BPDIR}\PY{p}{,}\PY{l+s+s1}{\PYZsq{}}\PY{l+s+s1}{3235.txt}\PY{l+s+s1}{\PYZsq{}}\PY{p}{)}\PY{p}{,} \PY{n}{header}\PY{o}{=}\PY{k+kc}{None}\PY{p}{,} \PY{n}{names}\PY{o}{=}\PY{p}{[}\PY{l+s+s2}{\PYZdq{}}\PY{l+s+s2}{systolic}\PY{l+s+s2}{\PYZdq{}}\PY{p}{,} \PY{l+s+s2}{\PYZdq{}}\PY{l+s+s2}{diastolic}\PY{l+s+s2}{\PYZdq{}}\PY{p}{,} \PY{p}{]}\PY{p}{)}
         \PY{n+nb}{print}\PY{p}{(}\PY{n}{bp}\PY{o}{.}\PY{n}{shape}\PY{p}{,} \PY{n}{bp}\PY{o}{.}\PY{n}{size}\PY{p}{)}
         \PY{n+nb}{print}\PY{p}{(}\PY{n}{bp}\PY{o}{.}\PY{n}{dtypes}\PY{p}{)}
         \PY{n}{bp}\PY{o}{.}\PY{n}{head}\PY{p}{(}\PY{p}{)}
\end{Verbatim}


    \begin{Verbatim}[commandchars=\\\{\}]
(485, 2) 970
systolic     int64
diastolic    int64
dtype: object

    \end{Verbatim}

\begin{Verbatim}[commandchars=\\\{\}]
{\color{outcolor}Out[{\color{outcolor}24}]:}    systolic  diastolic
         0       128         68
         1       128         69
         2       129         74
         3       100         54
         4       108         56
\end{Verbatim}
            
    \hypertarget{compute-summary-statistics}{%
\paragraph{Compute summary
statistics}\label{compute-summary-statistics}}

We can compute summary statistics on a Pandas DataFrame or Series using
either a numpy function or a method of the DataFrame or Series.

\hypertarget{how-do-we-know-what-numpy-functions-are-defined-or-what-methods-a-dataframe-has}{%
\subsubsection{How do we know what numpy functions are defined or what
methods a DataFrame
has?}\label{how-do-we-know-what-numpy-functions-are-defined-or-what-methods-a-dataframe-has}}

    \begin{Verbatim}[commandchars=\\\{\}]
{\color{incolor}In [{\color{incolor}25}]:} \PY{n+nb}{print}\PY{p}{(}\PY{n}{hr}\PY{o}{.}\PY{n}{max}\PY{p}{(}\PY{p}{)}\PY{p}{)}
\end{Verbatim}


    \begin{Verbatim}[commandchars=\\\{\}]
heart\_rate    303
dtype: int64

    \end{Verbatim}

    \begin{Verbatim}[commandchars=\\\{\}]
{\color{incolor}In [{\color{incolor}26}]:} \PY{n}{np}\PY{o}{.}\PY{n}{max}\PY{p}{(}\PY{n}{hr}\PY{p}{)}
\end{Verbatim}


\begin{Verbatim}[commandchars=\\\{\}]
{\color{outcolor}Out[{\color{outcolor}26}]:} heart\_rate    303
         dtype: int64
\end{Verbatim}
            
    \begin{Verbatim}[commandchars=\\\{\}]
{\color{incolor}In [{\color{incolor}27}]:} \PY{n}{np}\PY{o}{.}\PY{n}{max}\PY{p}{(}\PY{n}{hr}\PY{p}{[}\PY{l+s+s2}{\PYZdq{}}\PY{l+s+s2}{heart\PYZus{}rate}\PY{l+s+s2}{\PYZdq{}}\PY{p}{]}\PY{p}{)}
\end{Verbatim}


\begin{Verbatim}[commandchars=\\\{\}]
{\color{outcolor}Out[{\color{outcolor}27}]:} 303
\end{Verbatim}
            
    \begin{Verbatim}[commandchars=\\\{\}]
{\color{incolor}In [{\color{incolor}28}]:} \PY{n}{hr}\PY{p}{[}\PY{l+s+s2}{\PYZdq{}}\PY{l+s+s2}{heart\PYZus{}rate}\PY{l+s+s2}{\PYZdq{}}\PY{p}{]}\PY{o}{.}\PY{n}{max}\PY{p}{(}\PY{p}{)}
\end{Verbatim}


\begin{Verbatim}[commandchars=\\\{\}]
{\color{outcolor}Out[{\color{outcolor}28}]:} 303
\end{Verbatim}
            
    \hypertarget{exercise}{%
\subsection{Exercise}\label{exercise}}

What is the median value of the diastolic blood pressure for patient
3235?

    \begin{Verbatim}[commandchars=\\\{\}]
{\color{incolor}In [{\color{incolor}46}]:} \PY{n}{median\PYZus{}diastolic}\PY{p}{(}\PY{n}{bp}\PY{p}{[}\PY{l+s+s2}{\PYZdq{}}\PY{l+s+s2}{diastolic}\PY{l+s+s2}{\PYZdq{}}\PY{p}{]}\PY{o}{.}\PY{n}{median}\PY{p}{(}\PY{p}{)}\PY{p}{)}
         \PY{c+c1}{\PYZsh{}print(bp[\PYZdq{}diastolic\PYZdq{}].median())}
\end{Verbatim}


\begin{Verbatim}[commandchars=\\\{\}]
{\color{outcolor}Out[{\color{outcolor}46}]:} 'You provided the correct median value'
\end{Verbatim}
            
    \hypertarget{python-aside}{%
\subsubsection{Python Aside}\label{python-aside}}

It can often become confusing of what variables (including functions)
are alive and well in your workspace. Python provides some functions for
exploring this question:

\begin{itemize}
\tightlist
\item
  \texttt{dir()}
\item
  \texttt{globals()}
\item
  \texttt{locals()}
\end{itemize}

IPython provides magics for this \texttt{\%who} and \texttt{\%whos}

I find keeping track of variables in Jupyter notebooks more important
than in traditional scripting because I end up with a lot of global
variables.

    \begin{Verbatim}[commandchars=\\\{\}]
{\color{incolor}In [{\color{incolor}35}]:} \PY{o}{\PYZpc{}}\PY{k}{who}
\end{Verbatim}


    \begin{Verbatim}[commandchars=\\\{\}]
BPDIR	 DATADIR	 HRDIR	 assert\_equal	 bp	 data\_shape	 data\_types	 hr	 hr\_files	 
median\_diastolic	 np	 numbers	 os	 pd	 ra	 utils	 

    \end{Verbatim}

    \begin{Verbatim}[commandchars=\\\{\}]
{\color{incolor}In [{\color{incolor}36}]:} \PY{o}{\PYZpc{}}\PY{k}{whos}
\end{Verbatim}


    \begin{Verbatim}[commandchars=\\\{\}]
Variable           Type         Data/Info
-----------------------------------------
BPDIR              str          /home/achetam/DATA/Numerics/mimic2/bp/subjects
DATADIR            str          /home/achetam/DATA
HRDIR              str          /home/achetam/DATA/Numerics/mimic2/hr/subjects
assert\_equal       method       <bound method TestCase.as<{\ldots}>al.Dummy testMethod=nop>>
bp                 DataFrame         systolic  diastolic\textbackslash{}<{\ldots}>n\textbackslash{}n[485 rows x 2 columns]
data\_shape         function     <function data\_shape at 0x7f412c9530d0>
data\_types         function     <function data\_types at 0x7f412c953158>
hr                 DataFrame         heart\_rate\textbackslash{}n0       <{\ldots}>n\textbackslash{}n[483 rows x 1 columns]
hr\_files           list         n=3903
median\_diastolic   function     <function median\_diastolic at 0x7f412bc5fbf8>
np                 module       <module 'numpy' from '/op<{\ldots}>kages/numpy/\_\_init\_\_.py'>
numbers            module       <module 'numbers' from '/<{\ldots}>ib/python3.5/numbers.py'>
os                 module       <module 'os' from '/opt/c<{\ldots}>nda/lib/python3.5/os.py'>
pd                 module       <module 'pandas' from '/o<{\ldots}>ages/pandas/\_\_init\_\_.py'>
ra                 module       <module 'numpy.random' fr<{\ldots}>umpy/random/\_\_init\_\_.py'>
utils              module       <module 'utils' from '/ho<{\ldots}>ot\_camp\_2\_2018/utils.py'>

    \end{Verbatim}

    \hypertarget{exercise}{%
\subsection{Exercise}\label{exercise}}

To explore equality of floating point numbers, perturb your answer by
adding \texttt{0.1}, \texttt{0.01}, etc. until your perturbed answer is
considered equal.

    \begin{Verbatim}[commandchars=\\\{\}]
{\color{incolor}In [{\color{incolor}63}]:} \PY{n}{median\PYZus{}diastolic}\PY{p}{(}\PY{n}{bp}\PY{p}{[}\PY{l+s+s2}{\PYZdq{}}\PY{l+s+s2}{diastolic}\PY{l+s+s2}{\PYZdq{}}\PY{p}{]}\PY{o}{.}\PY{n}{median}\PY{p}{(}\PY{p}{)} \PY{o}{+} \PY{l+m+mf}{0.000000000000001}\PY{p}{)}
\end{Verbatim}


\begin{Verbatim}[commandchars=\\\{\}]
{\color{outcolor}Out[{\color{outcolor}63}]:} 'You provided the correct median value'
\end{Verbatim}
            
    \hypertarget{describe}{%
\subsection{\texorpdfstring{\href{https://pandas.pydata.org/pandas-docs/stable/generated/pandas.DataFrame.describe.html}{\texttt{describe()}}}{describe()}}\label{describe}}

Pandas DataFrames (Series) come with a \texttt{describe()} method that
provides summary statistics.

    \begin{Verbatim}[commandchars=\\\{\}]
{\color{incolor}In [{\color{incolor}64}]:} \PY{n}{hr}\PY{o}{.}\PY{n}{describe}\PY{p}{(}\PY{p}{)}
\end{Verbatim}


\begin{Verbatim}[commandchars=\\\{\}]
{\color{outcolor}Out[{\color{outcolor}64}]:}        heart\_rate
         count  483.000000
         mean    92.857143
         std     14.398532
         min     75.000000
         25\%     85.000000
         50\%     90.000000
         75\%     99.000000
         max    303.000000
\end{Verbatim}
            
    \begin{Verbatim}[commandchars=\\\{\}]
{\color{incolor}In [{\color{incolor}66}]:} \PY{n}{bp}\PY{o}{.}\PY{n}{describe}\PY{p}{(}\PY{p}{)}
\end{Verbatim}


\begin{Verbatim}[commandchars=\\\{\}]
{\color{outcolor}Out[{\color{outcolor}66}]:}          systolic   diastolic
         count  485.000000  485.000000
         mean   112.554639   53.513402
         std     15.861865    6.555416
         min     36.000000   31.000000
         25\%    101.000000   50.000000
         50\%    113.000000   54.000000
         75\%    122.000000   57.000000
         max    166.000000   79.000000
\end{Verbatim}
            
    \hypertarget{creating-a-new-column}{%
\subsection{Creating a New Column}\label{creating-a-new-column}}

We can create a new column in the DataFrame with a simple assignment
statement:

    \begin{Verbatim}[commandchars=\\\{\}]
{\color{incolor}In [{\color{incolor}67}]:} \PY{n}{hr}\PY{p}{[}\PY{l+s+s2}{\PYZdq{}}\PY{l+s+s2}{one}\PY{l+s+s2}{\PYZdq{}}\PY{p}{]} \PY{o}{=} \PY{l+m+mi}{1}
         \PY{n}{hr}\PY{p}{[}\PY{l+s+s2}{\PYZdq{}}\PY{l+s+s2}{range}\PY{l+s+s2}{\PYZdq{}}\PY{p}{]} \PY{o}{=} \PY{n+nb}{range}\PY{p}{(}\PY{n+nb}{len}\PY{p}{(}\PY{n}{hr}\PY{p}{)}\PY{p}{)}
         \PY{n}{hr}\PY{p}{[}\PY{l+s+s2}{\PYZdq{}}\PY{l+s+s2}{inverse range}\PY{l+s+s2}{\PYZdq{}}\PY{p}{]} \PY{o}{=} \PY{n+nb}{range}\PY{p}{(}\PY{n+nb}{len}\PY{p}{(}\PY{n}{hr}\PY{p}{)}\PY{p}{,}\PY{l+m+mi}{0}\PY{p}{,} \PY{o}{\PYZhy{}}\PY{l+m+mi}{1}\PY{p}{)}
         \PY{n}{hr}\PY{o}{.}\PY{n}{head}\PY{p}{(}\PY{p}{)}
\end{Verbatim}


\begin{Verbatim}[commandchars=\\\{\}]
{\color{outcolor}Out[{\color{outcolor}67}]:}    heart\_rate  one  range  inverse range
         0          88    1      0            483
         1          84    1      1            482
         2          87    1      2            481
         3          78    1      3            480
         4          85    1      4            479
\end{Verbatim}
            
    We can also create a new column based on a function of the existing
columns.

We can do this in two ways

\hypertarget{method-1}{%
\subsubsection{Method 1}\label{method-1}}

    \begin{Verbatim}[commandchars=\\\{\}]
{\color{incolor}In [{\color{incolor}68}]:} \PY{n}{hr}\PY{p}{[}\PY{l+s+s2}{\PYZdq{}}\PY{l+s+s2}{range\PYZus{}diff}\PY{l+s+s2}{\PYZdq{}}\PY{p}{]} \PY{o}{=} \PY{n}{hr}\PY{p}{[}\PY{l+s+s2}{\PYZdq{}}\PY{l+s+s2}{inverse range}\PY{l+s+s2}{\PYZdq{}}\PY{p}{]} \PY{o}{\PYZhy{}} \PY{n}{hr}\PY{p}{[}\PY{l+s+s2}{\PYZdq{}}\PY{l+s+s2}{range}\PY{l+s+s2}{\PYZdq{}}\PY{p}{]}
         \PY{n}{hr}\PY{o}{.}\PY{n}{head}\PY{p}{(}\PY{p}{)}
\end{Verbatim}


\begin{Verbatim}[commandchars=\\\{\}]
{\color{outcolor}Out[{\color{outcolor}68}]:}    heart\_rate  one  range  inverse range  range\_diff
         0          88    1      0            483         483
         1          84    1      1            482         481
         2          87    1      2            481         479
         3          78    1      3            480         477
         4          85    1      4            479         475
\end{Verbatim}
            
    \hypertarget{method-2}{%
\subsubsection{Method 2}\label{method-2}}

Our second method uses the \href{}{\texttt{apply()}} method and a
function. For this, I frequently use
\href{https://docs.python.org/3/tutorial/controlflow.html\#lambda-expressions}{\textbf{anonymous
function}}.

\hypertarget{anonymous-functions}{%
\paragraph{Anonymous Functions}\label{anonymous-functions}}

The syntax for anonymous functions is as follows:

\begin{Shaded}
\begin{Highlighting}[]
\KeywordTok{lambda}\NormalTok{ variable(s): some_function_of_the_variable(s)}
\end{Highlighting}
\end{Shaded}

So an anonymous doubling function would be

\begin{Shaded}
\begin{Highlighting}[]
\KeywordTok{lambda}\NormalTok{ x: }\DecValTok{2}\OperatorTok{*}\NormalTok{x}
\end{Highlighting}
\end{Shaded}

Here is an anonymous function that returns the upper case version of a
string

\begin{Shaded}
\begin{Highlighting}[]
\KeywordTok{lambda}\NormalTok{ y: y.upper()}
\end{Highlighting}
\end{Shaded}

\hypertarget{pandas-apply}{%
\paragraph{\texorpdfstring{Pandas
\texttt{apply()}}{Pandas apply()}}\label{pandas-apply}}

\begin{itemize}
\tightlist
\item
  \texttt{apply()} applies a function to each element in the DataFrame
  (Series)
\item
  We specify whether we want to apply the function by columns or rows by
  specifying the keyword argument \texttt{axis} (which defaults to 0 for
  columns)

  \begin{itemize}
  \tightlist
  \item
    Remembering to set \texttt{axis} is very important and something I
    frequently to do
  \end{itemize}
\item
  When we apply by rows, we get a variable that contains each row; we
  can access specific columns within the row
\end{itemize}

    \begin{Verbatim}[commandchars=\\\{\}]
{\color{incolor}In [{\color{incolor}69}]:} \PY{n}{hr}\PY{p}{[}\PY{l+s+s2}{\PYZdq{}}\PY{l+s+s2}{range\PYZus{}diff2}\PY{l+s+s2}{\PYZdq{}}\PY{p}{]} \PY{o}{=} \PY{n}{hr}\PY{o}{.}\PY{n}{apply}\PY{p}{(}\PY{k}{lambda}  \PY{n}{row}\PY{p}{:} \PY{n}{row}\PY{p}{[}\PY{l+s+s2}{\PYZdq{}}\PY{l+s+s2}{range\PYZus{}diff}\PY{l+s+s2}{\PYZdq{}}\PY{p}{]} \PY{o}{\PYZhy{}} \PY{n}{row}\PY{p}{[}\PY{l+s+s2}{\PYZdq{}}\PY{l+s+s2}{range}\PY{l+s+s2}{\PYZdq{}}\PY{p}{]}\PY{p}{,} 
                                      \PY{n}{axis}\PY{o}{=}\PY{l+m+mi}{1}\PY{p}{)}
         \PY{n}{hr}\PY{o}{.}\PY{n}{head}\PY{p}{(}\PY{p}{)}
\end{Verbatim}


\begin{Verbatim}[commandchars=\\\{\}]
{\color{outcolor}Out[{\color{outcolor}69}]:}    heart\_rate  one  range  inverse range  range\_diff  range\_diff2
         0          88    1      0            483         483          483
         1          84    1      1            482         481          480
         2          87    1      2            481         479          477
         3          78    1      3            480         477          474
         4          85    1      4            479         475          471
\end{Verbatim}
            
    \hypertarget{exercise}{%
\subsection{Exercise:}\label{exercise}}

\href{http://scikit-learn.org/stable/modules/preprocessing.html}{Data
standardization} transforms numeric data (\(x\)) by subtracting the mean
(\(\mu_x\)) and dividing by the standard deviation (\(\sigma_x\)):

\[\tilde{x} = \frac{x-\mu_x}{\sigma_x}\]

Read in one of the heart rate data files and compute the standardized
form of the data. Assign this normalized heart rate to the DataFrame
with a label of \textbf{``normalized hr''}.

    \begin{Verbatim}[commandchars=\\\{\}]
{\color{incolor}In [{\color{incolor}73}]:} \PY{n}{hr} \PY{o}{=} \PY{n}{pd}\PY{o}{.}\PY{n}{read\PYZus{}table}\PY{p}{(}\PY{n}{os}\PY{o}{.}\PY{n}{path}\PY{o}{.}\PY{n}{join}\PY{p}{(}\PY{n}{HRDIR}\PY{p}{,}\PY{l+s+s1}{\PYZsq{}}\PY{l+s+s1}{3236.txt}\PY{l+s+s1}{\PYZsq{}}\PY{p}{)}\PY{p}{,} \PY{n}{header}\PY{o}{=}\PY{k+kc}{None}\PY{p}{,} \PY{n}{names}\PY{o}{=}\PY{p}{[}\PY{l+s+s2}{\PYZdq{}}\PY{l+s+s2}{heart\PYZus{}rate}\PY{l+s+s2}{\PYZdq{}}\PY{p}{,}\PY{p}{]}\PY{p}{)}
         \PY{n}{hr}\PY{o}{.}\PY{n}{head}\PY{p}{(}\PY{p}{)}
\end{Verbatim}


\begin{Verbatim}[commandchars=\\\{\}]
{\color{outcolor}Out[{\color{outcolor}73}]:}    heart\_rate
         0          83
         1          77
         2          86
         3          82
         4          81
\end{Verbatim}
            
    \begin{Verbatim}[commandchars=\\\{\}]
{\color{incolor}In [{\color{incolor}81}]:} \PY{n}{hr}\PY{p}{[}\PY{l+s+s2}{\PYZdq{}}\PY{l+s+s2}{standardized\PYZus{}hr}\PY{l+s+s2}{\PYZdq{}}\PY{p}{]} \PY{o}{=} \PY{p}{(}\PY{n}{hr}\PY{p}{[}\PY{l+s+s2}{\PYZdq{}}\PY{l+s+s2}{heart\PYZus{}rate}\PY{l+s+s2}{\PYZdq{}}\PY{p}{]} \PY{o}{\PYZhy{}} \PY{n}{hr}\PY{p}{[}\PY{l+s+s2}{\PYZdq{}}\PY{l+s+s2}{heart\PYZus{}rate}\PY{l+s+s2}{\PYZdq{}}\PY{p}{]}\PY{o}{.}\PY{n}{mean}\PY{p}{(}\PY{p}{)}\PY{p}{)} \PY{o}{/} \PY{n}{hr}\PY{p}{[}\PY{l+s+s2}{\PYZdq{}}\PY{l+s+s2}{heart\PYZus{}rate}\PY{l+s+s2}{\PYZdq{}}\PY{p}{]}\PY{o}{.}\PY{n}{std}\PY{p}{(}\PY{p}{)}
         \PY{n}{hr}\PY{o}{.}\PY{n}{head}\PY{p}{(}\PY{p}{)}
\end{Verbatim}


\begin{Verbatim}[commandchars=\\\{\}]
{\color{outcolor}Out[{\color{outcolor}81}]:}    heart\_rate  standardized  standardized\_hr
         0          83     -0.066635        -0.066635
         1          77     -0.899576        -0.899576
         2          86      0.349835         0.349835
         3          82     -0.205459        -0.205459
         4          81     -0.344282        -0.344282
\end{Verbatim}
            
    \begin{Verbatim}[commandchars=\\\{\}]
{\color{incolor}In [{\color{incolor}85}]:} \PY{n}{hr}\PY{p}{[}\PY{l+s+s2}{\PYZdq{}}\PY{l+s+s2}{heart\PYZus{}rate}\PY{l+s+s2}{\PYZdq{}}\PY{p}{]}\PY{o}{.}\PY{n}{mean}\PY{p}{(}\PY{p}{)}
         \PY{n}{hr}\PY{p}{[}\PY{l+s+s2}{\PYZdq{}}\PY{l+s+s2}{heart\PYZus{}rate}\PY{l+s+s2}{\PYZdq{}}\PY{p}{]}\PY{o}{.}\PY{n}{std}\PY{p}{(}\PY{p}{)}
         \PY{n}{hr}\PY{o}{.}\PY{n}{drop}\PY{p}{(}\PY{p}{[}\PY{l+s+s2}{\PYZdq{}}\PY{l+s+s2}{standardized}\PY{l+s+s2}{\PYZdq{}}\PY{p}{]}\PY{p}{,} \PY{n}{axis} \PY{o}{=} \PY{l+m+mi}{1}\PY{p}{)}
\end{Verbatim}


\begin{Verbatim}[commandchars=\\\{\}]
{\color{outcolor}Out[{\color{outcolor}85}]:}      heart\_rate  standardized\_hr
         0            83        -0.066635
         1            77        -0.899576
         2            86         0.349835
         3            82        -0.205459
         4            81        -0.344282
         5            84         0.072188
         6            83        -0.066635
         7            85         0.211012
         8            85         0.211012
         9            87         0.488659
         10           88         0.627482
         11           90         0.905129
         12           88         0.627482
         13           88         0.627482
         14           86         0.349835
         15           85         0.211012
         16           83        -0.066635
         17           81        -0.344282
         18           80        -0.483106
         19           80        -0.483106
         20           82        -0.205459
         21           80        -0.483106
         22           78        -0.760753
         23           80        -0.483106
         24           71        -1.732517
         25           66        -2.426634
         26           68        -2.148988
         27           71        -1.732517
         28           72        -1.593694
         29           92         1.182776
         ..          {\ldots}              {\ldots}
         120          81        -0.344282
         121          77        -0.899576
         122          81        -0.344282
         123          82        -0.205459
         124          87         0.488659
         125          88         0.627482
         126          87         0.488659
         127          98         2.015717
         128          89         0.766306
         129          84         0.072188
         130          84         0.072188
         131          83        -0.066635
         132          86         0.349835
         133          90         0.905129
         134          92         1.182776
         135          90         0.905129
         136          92         1.182776
         137          88         0.627482
         138          87         0.488659
         139          90         0.905129
         140          84         0.072188
         141          82        -0.205459
         142          84         0.072188
         143          86         0.349835
         144          78        -0.760753
         145          78        -0.760753
         146          79        -0.621929
         147          73        -1.454870
         148          70        -1.871341
         149          77        -0.899576
         
         [150 rows x 2 columns]
\end{Verbatim}
            
    \hypertarget{exercise}{%
\subsection{Exercise}\label{exercise}}

The student's t-test is often used to comapre two populations. Using
\texttt{len}, \texttt{math.sqrt} and numpy methods/functions compute the
\href{https://en.wikipedia.org/wiki/Student\%27s_t-test\#Independent_two-sample_t-test}{t
value} for two files.

\[t = \frac{\bar{{X_1}}-\bar{X_2}}{s_{X_1X_2}\sqrt{\frac{1}{n_1}+\frac{1}{n_2}}}\]
where
\[s_{X_1X_2} = \sqrt{\frac{(n_1-1)s_{X_1}^2 + (n_2-1)s_{X_2}^2}{n_1+n_2-2}}\].
1. Pick two heart rate files and comptue the t value for the two files.

    \begin{Verbatim}[commandchars=\\\{\}]
{\color{incolor}In [{\color{incolor}91}]:} \PY{n}{hr2} \PY{o}{=} \PY{n}{hr} \PY{o}{=} \PY{n}{pd}\PY{o}{.}\PY{n}{read\PYZus{}table}\PY{p}{(}\PY{n}{os}\PY{o}{.}\PY{n}{path}\PY{o}{.}\PY{n}{join}\PY{p}{(}\PY{n}{HRDIR}\PY{p}{,}\PY{l+s+s1}{\PYZsq{}}\PY{l+s+s1}{3220.txt}\PY{l+s+s1}{\PYZsq{}}\PY{p}{)}\PY{p}{,} \PY{n}{header}\PY{o}{=}\PY{k+kc}{None}\PY{p}{,} \PY{n}{names}\PY{o}{=}\PY{p}{[}\PY{l+s+s2}{\PYZdq{}}\PY{l+s+s2}{heart\PYZus{}rate}\PY{l+s+s2}{\PYZdq{}}\PY{p}{,}\PY{p}{]}\PY{p}{)}
         \PY{n}{hr}\PY{o}{.}\PY{n}{head}\PY{p}{(}\PY{p}{)}
\end{Verbatim}


    \begin{Verbatim}[commandchars=\\\{\}]

        ---------------------------------------------------------------------------

        FileNotFoundError                         Traceback (most recent call last)

        <ipython-input-91-fe851ac97fd7> in <module>()
    ----> 1 hr2 = hr = pd.read\_table(os.path.join(HRDIR,'3228.txt'), header=None, names=["heart\_rate",])
          2 hr.head()


        /opt/conda/lib/python3.5/site-packages/pandas/io/parsers.py in parser\_f(filepath\_or\_buffer, sep, delimiter, header, names, index\_col, usecols, squeeze, prefix, mangle\_dupe\_cols, dtype, engine, converters, true\_values, false\_values, skipinitialspace, skiprows, nrows, na\_values, keep\_default\_na, na\_filter, verbose, skip\_blank\_lines, parse\_dates, infer\_datetime\_format, keep\_date\_col, date\_parser, dayfirst, iterator, chunksize, compression, thousands, decimal, lineterminator, quotechar, quoting, escapechar, comment, encoding, dialect, tupleize\_cols, error\_bad\_lines, warn\_bad\_lines, skipfooter, doublequote, delim\_whitespace, low\_memory, memory\_map, float\_precision)
        676                     skip\_blank\_lines=skip\_blank\_lines)
        677 
    --> 678         return \_read(filepath\_or\_buffer, kwds)
        679 
        680     parser\_f.\_\_name\_\_ = name


        /opt/conda/lib/python3.5/site-packages/pandas/io/parsers.py in \_read(filepath\_or\_buffer, kwds)
        438 
        439     \# Create the parser.
    --> 440     parser = TextFileReader(filepath\_or\_buffer, **kwds)
        441 
        442     if chunksize or iterator:


        /opt/conda/lib/python3.5/site-packages/pandas/io/parsers.py in \_\_init\_\_(self, f, engine, **kwds)
        785             self.options['has\_index\_names'] = kwds['has\_index\_names']
        786 
    --> 787         self.\_make\_engine(self.engine)
        788 
        789     def close(self):


        /opt/conda/lib/python3.5/site-packages/pandas/io/parsers.py in \_make\_engine(self, engine)
       1012     def \_make\_engine(self, engine='c'):
       1013         if engine == 'c':
    -> 1014             self.\_engine = CParserWrapper(self.f, **self.options)
       1015         else:
       1016             if engine == 'python':


        /opt/conda/lib/python3.5/site-packages/pandas/io/parsers.py in \_\_init\_\_(self, src, **kwds)
       1706         kwds['usecols'] = self.usecols
       1707 
    -> 1708         self.\_reader = parsers.TextReader(src, **kwds)
       1709 
       1710         passed\_names = self.names is None


        pandas/\_libs/parsers.pyx in pandas.\_libs.parsers.TextReader.\_\_cinit\_\_()


        pandas/\_libs/parsers.pyx in pandas.\_libs.parsers.TextReader.\_setup\_parser\_source()


        FileNotFoundError: File b'/home/achetam/DATA/Numerics/mimic2/hr/subjects/3228.txt' does not exist

    \end{Verbatim}

    \begin{Verbatim}[commandchars=\\\{\}]
{\color{incolor}In [{\color{incolor}90}]:} \PY{n}{hr2}\PY{p}{[}\PY{l+s+s2}{\PYZdq{}}\PY{l+s+s2}{standardized\PYZus{}hr}\PY{l+s+s2}{\PYZdq{}}\PY{p}{]} \PY{o}{=} \PY{p}{(}\PY{n}{hr2}\PY{p}{[}\PY{l+s+s2}{\PYZdq{}}\PY{l+s+s2}{heart\PYZus{}rate}\PY{l+s+s2}{\PYZdq{}}\PY{p}{]} \PY{o}{\PYZhy{}} \PY{n}{hr2}\PY{p}{[}\PY{l+s+s2}{\PYZdq{}}\PY{l+s+s2}{heart\PYZus{}rate}\PY{l+s+s2}{\PYZdq{}}\PY{p}{]}\PY{o}{.}\PY{n}{mean}\PY{p}{(}\PY{p}{)}\PY{p}{)} \PY{o}{/} \PY{n}{hr2}\PY{p}{[}\PY{l+s+s2}{\PYZdq{}}\PY{l+s+s2}{heart\PYZus{}rate}\PY{l+s+s2}{\PYZdq{}}\PY{p}{]}\PY{o}{.}\PY{n}{std}\PY{p}{(}\PY{p}{)}
         \PY{n}{hr2}\PY{o}{.}\PY{n}{head}\PY{p}{(}\PY{p}{)}
\end{Verbatim}


\begin{Verbatim}[commandchars=\\\{\}]
{\color{outcolor}Out[{\color{outcolor}90}]:}    heart\_rate  standardized\_hr
         0          99         1.052707
         1         102         1.487272
         2         102         1.487272
         3         101         1.342417
         4         100         1.197562
\end{Verbatim}
            
    {University of Uah Data Science for Health} by {Brian E. Chapman} is
licensed under a Creative Commons Attribution-NonCommercial-ShareAlike
4.0 International License.


    % Add a bibliography block to the postdoc
    
    
    
    \end{document}
